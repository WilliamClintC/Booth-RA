% Options for packages loaded elsewhere
\PassOptionsToPackage{unicode}{hyperref}
\PassOptionsToPackage{hyphens}{url}
\PassOptionsToPackage{dvipsnames,svgnames,x11names}{xcolor}
%
\documentclass[
  letterpaper,
  DIV=11,
  numbers=noendperiod]{scrartcl}

\usepackage{amsmath,amssymb}
\usepackage{iftex}
\ifPDFTeX
  \usepackage[T1]{fontenc}
  \usepackage[utf8]{inputenc}
  \usepackage{textcomp} % provide euro and other symbols
\else % if luatex or xetex
  \usepackage{unicode-math}
  \defaultfontfeatures{Scale=MatchLowercase}
  \defaultfontfeatures[\rmfamily]{Ligatures=TeX,Scale=1}
\fi
\usepackage{lmodern}
\ifPDFTeX\else  
    % xetex/luatex font selection
\fi
% Use upquote if available, for straight quotes in verbatim environments
\IfFileExists{upquote.sty}{\usepackage{upquote}}{}
\IfFileExists{microtype.sty}{% use microtype if available
  \usepackage[]{microtype}
  \UseMicrotypeSet[protrusion]{basicmath} % disable protrusion for tt fonts
}{}
\makeatletter
\@ifundefined{KOMAClassName}{% if non-KOMA class
  \IfFileExists{parskip.sty}{%
    \usepackage{parskip}
  }{% else
    \setlength{\parindent}{0pt}
    \setlength{\parskip}{6pt plus 2pt minus 1pt}}
}{% if KOMA class
  \KOMAoptions{parskip=half}}
\makeatother
\usepackage{xcolor}
\usepackage[margin=1in]{geometry}
\setlength{\emergencystretch}{3em} % prevent overfull lines
\setcounter{secnumdepth}{5}
% Make \paragraph and \subparagraph free-standing
\makeatletter
\ifx\paragraph\undefined\else
  \let\oldparagraph\paragraph
  \renewcommand{\paragraph}{
    \@ifstar
      \xxxParagraphStar
      \xxxParagraphNoStar
  }
  \newcommand{\xxxParagraphStar}[1]{\oldparagraph*{#1}\mbox{}}
  \newcommand{\xxxParagraphNoStar}[1]{\oldparagraph{#1}\mbox{}}
\fi
\ifx\subparagraph\undefined\else
  \let\oldsubparagraph\subparagraph
  \renewcommand{\subparagraph}{
    \@ifstar
      \xxxSubParagraphStar
      \xxxSubParagraphNoStar
  }
  \newcommand{\xxxSubParagraphStar}[1]{\oldsubparagraph*{#1}\mbox{}}
  \newcommand{\xxxSubParagraphNoStar}[1]{\oldsubparagraph{#1}\mbox{}}
\fi
\makeatother


\providecommand{\tightlist}{%
  \setlength{\itemsep}{0pt}\setlength{\parskip}{0pt}}\usepackage{longtable,booktabs,array}
\usepackage{calc} % for calculating minipage widths
% Correct order of tables after \paragraph or \subparagraph
\usepackage{etoolbox}
\makeatletter
\patchcmd\longtable{\par}{\if@noskipsec\mbox{}\fi\par}{}{}
\makeatother
% Allow footnotes in longtable head/foot
\IfFileExists{footnotehyper.sty}{\usepackage{footnotehyper}}{\usepackage{footnote}}
\makesavenoteenv{longtable}
\usepackage{graphicx}
\makeatletter
\def\maxwidth{\ifdim\Gin@nat@width>\linewidth\linewidth\else\Gin@nat@width\fi}
\def\maxheight{\ifdim\Gin@nat@height>\textheight\textheight\else\Gin@nat@height\fi}
\makeatother
% Scale images if necessary, so that they will not overflow the page
% margins by default, and it is still possible to overwrite the defaults
% using explicit options in \includegraphics[width, height, ...]{}
\setkeys{Gin}{width=\maxwidth,height=\maxheight,keepaspectratio}
% Set default figure placement to htbp
\makeatletter
\def\fps@figure{htbp}
\makeatother

\KOMAoption{captions}{tableheading}
\makeatletter
\@ifpackageloaded{caption}{}{\usepackage{caption}}
\AtBeginDocument{%
\ifdefined\contentsname
  \renewcommand*\contentsname{Table of contents}
\else
  \newcommand\contentsname{Table of contents}
\fi
\ifdefined\listfigurename
  \renewcommand*\listfigurename{List of Figures}
\else
  \newcommand\listfigurename{List of Figures}
\fi
\ifdefined\listtablename
  \renewcommand*\listtablename{List of Tables}
\else
  \newcommand\listtablename{List of Tables}
\fi
\ifdefined\figurename
  \renewcommand*\figurename{Figure}
\else
  \newcommand\figurename{Figure}
\fi
\ifdefined\tablename
  \renewcommand*\tablename{Table}
\else
  \newcommand\tablename{Table}
\fi
}
\@ifpackageloaded{float}{}{\usepackage{float}}
\floatstyle{ruled}
\@ifundefined{c@chapter}{\newfloat{codelisting}{h}{lop}}{\newfloat{codelisting}{h}{lop}[chapter]}
\floatname{codelisting}{Listing}
\newcommand*\listoflistings{\listof{codelisting}{List of Listings}}
\makeatother
\makeatletter
\makeatother
\makeatletter
\@ifpackageloaded{caption}{}{\usepackage{caption}}
\@ifpackageloaded{subcaption}{}{\usepackage{subcaption}}
\makeatother
\ifLuaTeX
  \usepackage{selnolig}  % disable illegal ligatures
\fi
\usepackage{bookmark}

\IfFileExists{xurl.sty}{\usepackage{xurl}}{} % add URL line breaks if available
\urlstyle{same} % disable monospaced font for URLs
\hypersetup{
  pdftitle={Geocoding Truck Stops Documentation},
  pdfauthor={William Co},
  pdfkeywords={sample, academic, report, quarto},
  colorlinks=true,
  linkcolor={blue},
  filecolor={Maroon},
  citecolor={Blue},
  urlcolor={Blue},
  pdfcreator={LaTeX via pandoc}}

\title{Geocoding Truck Stops Documentation}
\author{William Co}
\date{2025-08-27}

\begin{document}
\maketitle
\begin{abstract}
This report documents the geocoding of U.S. truck stop data, addressing
challenges from inconsistent address formats. Using phone number
matching and structured data from Truck Stops and Services, Yelp, Yellow
Pages, and iExit, we achieved a 99.19\% match rate. A custom interface
was developed to support manual verification and ensure data accuracy.
\end{abstract}

\renewcommand*\contentsname{Table of contents}
{
\hypersetup{linkcolor=}
\setcounter{tocdepth}{3}
\tableofcontents
}
\section{Data Cleaning}\label{data-cleaning}

\subsection{Shape and Column Analysis}\label{shape-and-column-analysis}

We begin by checking the structure and columns of the dataset to ensure
consistency. The data contains 47,776 rows and 12 columns, including
variables such as weight, year, age, education, race, asset\_total,
asset\_housing, debt\_total, debt\_housing, and wealth. For our
analysis, we focus on the variables relevant to wealth and asset
calculations, and note that sex and income are not used further.

\begin{longtable}[]{@{}lll@{}}
\toprule\noalign{}
Column Name & Type & Description \\
\midrule\noalign{}
\endhead
\bottomrule\noalign{}
\endlastfoot
weight & float64 & Survey weight \\
year & int64 & Survey year \\
age & int64 & Age of respondent \\
sex & object & Sex (not used in analysis) \\
education & object & Education level \\
race & object & Race/ethnicity \\
asset\_total & float64 & Total assets \\
asset\_housing & float64 & Housing assets \\
debt\_total & float64 & Total debts \\
debt\_housing & float64 & Housing debts \\
income & float64 & Income (not used in analysis) \\
wealth & float64 & Calculated wealth \\
\end{longtable}

\subsection{Missing Values Check}\label{missing-values-check}

A review of the dataset shows that there are no missing values in any
column, so no imputation or removal of rows is necessary.

\subsection{Data Types and Ranges}\label{data-types-and-ranges}

Below are the observed data types and value ranges:

\begin{longtable}[]{@{}llll@{}}
\toprule\noalign{}
Variable & Type & Min & Max \\
\midrule\noalign{}
\endhead
\bottomrule\noalign{}
\endlastfoot
weight & float64 & 0.20 & 31,115.82 \\
year & int64 & 1989 & 2016 \\
age & int64 & 17 & 95 \\
sex & object & 2 unique & \\
education & object & 3 unique & \\
race & object & 4 unique & \\
asset\_total & float64 & -22,487,306.62 & 2,928,346,179.67 \\
asset\_housing & float64 & 0.00 & 182,642,128.63 \\
debt\_total & float64 & 0.00 & 293,486,997.64 \\
debt\_housing & float64 & 0.00 & 44,821,081.33 \\
income & float64 & 0.00 & 351,958,858.31 \\
wealth & float64 & -221,985,489.24 & 2,929,687,834.52 \\
\end{longtable}

\subsection{Negative Values Check and
Cleaning}\label{negative-values-check-and-cleaning}

We identify that asset\_total contains 7 negative values, which is about
0.01\% of the data. Since assets cannot logically be negative, we set
all negative values in asset\_total to zero. This adjustment ensures
that all asset values are non-negative, as required by financial logic.
After this cleaning step, asset\_total has a minimum value of zero, and
no negative values remain.

\begin{longtable}[]{@{}lll@{}}
\toprule\noalign{}
Variable & Negative Values & \% of Total \\
\midrule\noalign{}
\endhead
\bottomrule\noalign{}
\endlastfoot
weight & 0 & 0.00\% \\
asset\_total & 7 (before) & 0.01\% \\
asset\_total & 0 (after) & 0.00\% \\
asset\_housing & 0 & 0.00\% \\
debt\_total & 0 & 0.00\% \\
debt\_housing & 0 & 0.00\% \\
income & 0 & 0.00\% \\
\end{longtable}

A table of the rows with negative asset\_total values (before cleaning)
is available in the appendix or supplementary materials.

A summary table of asset\_total after cleaning:

\begin{longtable}[]{@{}ll@{}}
\toprule\noalign{}
Statistic & asset\_total \\
\midrule\noalign{}
\endhead
\bottomrule\noalign{}
\endlastfoot
Min & 0 \\
Max & 2,928,346,179.67 \\
Negative Values & 0 \\
\end{longtable}

\subsection{Outlier Detection}\label{outlier-detection}

We also check for outliers using the interquartile range (IQR) method.
While some variables have a notable number of outliers, these are
retained for analysis unless they are logically impossible (such as
negative assets, which have already been addressed).

\begin{longtable}[]{@{}
  >{\raggedright\arraybackslash}p{(\columnwidth - 8\tabcolsep) * \real{0.2394}}
  >{\raggedright\arraybackslash}p{(\columnwidth - 8\tabcolsep) * \real{0.1831}}
  >{\raggedright\arraybackslash}p{(\columnwidth - 8\tabcolsep) * \real{0.1690}}
  >{\raggedright\arraybackslash}p{(\columnwidth - 8\tabcolsep) * \real{0.2113}}
  >{\raggedright\arraybackslash}p{(\columnwidth - 8\tabcolsep) * \real{0.1972}}@{}}
\toprule\noalign{}
\begin{minipage}[b]{\linewidth}\raggedright
Variable
\end{minipage} & \begin{minipage}[b]{\linewidth}\raggedright
Outliers (N)
\end{minipage} & \begin{minipage}[b]{\linewidth}\raggedright
\% of Total
\end{minipage} & \begin{minipage}[b]{\linewidth}\raggedright
Lower Bound
\end{minipage} & \begin{minipage}[b]{\linewidth}\raggedright
Upper Bound
\end{minipage} \\
\midrule\noalign{}
\endhead
\bottomrule\noalign{}
\endlastfoot
weight & 330 & 0.7\% & -4,095 & 12,858 \\
asset\_total & 8,281 & 17.3\% & -2,215,818 & 3,831,215 \\
asset\_housing & 5,405 & 11.3\% & -651,383 & 1,085,639 \\
debt\_total & 5,091 & 10.7\% & -236,639 & 394,398 \\
debt\_housing & 5,033 & 10.5\% & -167,927 & 279,879 \\
income & 7,542 & 15.8\% & -179,464 & 385,518 \\
\end{longtable}

\subsection{Categorical Distribution}\label{categorical-distribution}

\begin{longtable}[]{@{}lll@{}}
\toprule\noalign{}
Race & Count & \% \\
\midrule\noalign{}
\endhead
\bottomrule\noalign{}
\endlastfoot
white & 37,044 & 77.5\% \\
black & 5,186 & 10.9\% \\
Hispanic & 3,553 & 7.4\% \\
other & 1,993 & 4.2\% \\
\end{longtable}

\begin{longtable}[]{@{}lll@{}}
\toprule\noalign{}
Education & Count & \% \\
\midrule\noalign{}
\endhead
\bottomrule\noalign{}
\endlastfoot
college degree & 19,444 & 40.7\% \\
no college & 17,820 & 37.3\% \\
some college & 10,512 & 22.0\% \\
\end{longtable}

\begin{longtable}[]{@{}lll@{}}
\toprule\noalign{}
Sex & Count & \% \\
\midrule\noalign{}
\endhead
\bottomrule\noalign{}
\endlastfoot
male & 37,212 & 77.9\% \\
female & 10,564 & 22.1\% \\
\end{longtable}

\subsection{Year Distribution}\label{year-distribution}

\begin{longtable}[]{@{}ll@{}}
\toprule\noalign{}
Year & Count \\
\midrule\noalign{}
\endhead
\bottomrule\noalign{}
\endlastfoot
1989 & 3,143 \\
1992 & 3,906 \\
1995 & 4,299 \\
1998 & 4,305 \\
2001 & 4,442 \\
2004 & 4,519 \\
2007 & 4,417 \\
2010 & 6,482 \\
2013 & 6,015 \\
2016 & 6,248 \\
\end{longtable}

\subsection{Wealth Variable}\label{wealth-variable}

We define the wealth variable using the following formula:

\[
\text{wealth} = \text{asset\_total} + \text{asset\_housing} - \text{debt\_total} - \text{debt\_housing}
\]

This formula is applied after cleaning asset\_total, ensuring that all
asset values used in the calculation are non-negative and logically
consistent for further analysis.

\subsection{Weight Variable}\label{weight-variable}

the weights are used in the following way. The weighted median is the
value ( m ) such that the sum of the weights of all observations less
than or equal to ( m ) is at least half the total weight, and the sum of
the weights of all observations greater than or equal to ( m ) is also
at least half the total weight.

Formally:

Let ( x\_1, x\_2, \ldots, x\_n ) be the sorted data values, and ( w\_1,
w\_2, \ldots, w\_n ) their corresponding weights.

Find the smallest ( m ) such that: {[} \sum\emph{\{x\_i \leq m\} w\_i
\geq \frac{1}{2} \sum}\{i=1\}\^{}n w\_i {]}

\section{1. Please summarize key trends in median total wealth over the
last 30 years by race and education using plots and in
writing.}\label{please-summarize-key-trends-in-median-total-wealth-over-the-last-30-years-by-race-and-education-using-plots-and-in-writing.}



\end{document}
